\documentclass[a4paper,12pt]{article}
\usepackage[utf8x]{inputenc}
\usepackage{amssymb}
\usepackage{amsfonts}
\usepackage{mathrsfs}
\usepackage{amsmath}
\usepackage{amsthm}
\usepackage[margin=3cm]{geometry}
\usepackage{times}
\usepackage{graphicx}
\usepackage{enumitem}
\usepackage{fancyhdr}
\usepackage{hyperref}
\usepackage{setspace}
\usepackage{subcaption}
\usepackage{mathtools}

% \usepackage{lineno}
% \linenumbers
% \renewcommand{\baselinestretch}{1.5}
% \usepackage{authblk}

\newtheorem{theorem}{Theorem}
\newtheorem{proposition}{Proposition}[section]
\newtheorem{lemma}{Lemma}[section]
\newtheorem{corollary}{Corollary}[section]
\theoremstyle{definition}
\newtheorem{definition}{Definition}[section]

\newcommand{\boldnabla}{\mbox{\boldmath$\nabla$}} % to be used in mathmode
\newcommand{\cbar}{\overline{\mathbb{C}}}% to be used in mathmode
\newcommand{\diff}[2]{\frac{d #1}{d #2}}% to be used in mathmode
\newcommand{\difff}[2]{\frac{d^2 #1}{d #2^2}}% to be used in mathmode
\newcommand{\pdiff}[2]{\frac{\partial #1}{\partial #2}} % to be used in mathmode
\newcommand{\pdifff}[2]{\frac{\partial^2 #1}{\partial #2^2}}% to be used in mathmode
\newcommand{\upperth}{$^{\mbox{\footnotesize{th}}}$}%to be used in text mode
\newcommand{\vect}[1]{\mathbf{#1}}% to be used in mathmode
\newcommand{\curl}[1]{\boldnabla \times \vect{#1}} % to be used in mathmode
\newcommand{\divr}[1]{\boldnabla \cdot \vect{#1}} %to be used in mathmode
\newcommand{\modu}[1]{\left| #1 \right|} %to be used in mathmode
\newcommand{\brak}[1]{\left( #1 \right)} % to be used in mathmode
\newcommand{\comm}[2]{\left[ #1 , #2 \right]} %to be used in mathmode
\newcommand{\dop}{\vect{d}} %to be used in mathmode
\newcommand{\cov}{\text{cov}} %to be used in mathmode
\newcommand{\var}{\text{var}} %to be used in mathmode
\newcommand{\mb}{\mathbf} %to be used in mathmode
\newcommand{\bs}{\boldsymbol} %to be used in mathmode
% Title Page
% \title{A simple two parameter distribution for modelling neuronal activity and capturing neuronal association}
% \date{}
%
% \author[1]{Thomas Delaney}
% \author[1]{Cian O'Donnell}
% \affil[1]{School of Computer Science, Electrical and Electronic Engineering, and Engineering Mathematics, University of Bristol, Bristol, United Kingdom.}
% \renewcommand\Affilfont{\itshape\small}

\begin{document}

\begin{description}
  \item[Title] Spike-Train Communities: Finding Groups of Similar Spike Trains
  \item[Author] Mark D. Humphries
  \item[Abstract]

    \begin{itemize}
      \item Determines the the maximum number of groups in a collection of spike trains. Groups the trains according to similarity.
      \item New insights into the encoding of aversive stimuli by dopaminergic neurons.
      \item The existence of neural ensembles that evolve in membership and characteristic timescale of organization during global slow oscillations.
    \end{itemize}

  \item[Introduction] There is an intro.

  \item[Materials \& Methods]
    \begin{itemize}
      \item The clustering algorithm
      \item Synthetic spike-train data for assessing clustering
      \item Neurophysiological data
    \end{itemize}

  \item[Results]
    \begin{itemize}
      \item The algorithm reliably and robustly finds repeating spike patterns.
      \item Differentiating hidden SNc cell responses to pain stimuli.
      \item No false positive groupings for SNc responses to control stimuli.
      \item Baseline response to pain stimulation drifts over experimental session.
      \item Postbicuculline response to pain stimulation is unexpectedly reliable.
      \item Detecting groups in simultaneous multineuron recordings.
      \item Correlation structure of cat V2 spike trains evolves under anesthesia.
      \item Transient correlation structure of cat V1 spike trains.
      \item The detection of large timescale structure in large datasets.
    \end{itemize}

\end{description}

\subsubsection*{Abbreviations}
\begin{description}
  \item[SNc] Substantia Nigra pars compacta
  \item[]
\end{description}

\end{document}
