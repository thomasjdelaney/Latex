\documentclass[a4paper,12pt]{article}
\usepackage[utf8x]{inputenc}
\usepackage{amssymb}
\usepackage{amsfonts}
\usepackage{mathrsfs}
\usepackage{amsmath}
\usepackage{amsthm}
\usepackage[margin=3cm]{geometry}
\usepackage{times}
\usepackage{graphicx}
\usepackage{dsfont}
\usepackage{enumitem}
\usepackage{fancyhdr} 
\usepackage{hyperref}
\usepackage{setspace}
\usepackage{gensymb}

\pagestyle{fancy}
\fancyhf{}
\lhead{Thomas Delaney}
\rhead{K\"{o}ster et al 2014: Notes}
\cfoot{\thepage}

\newtheorem{theorem}{Theorem}
\newtheorem{proposition}{Proposition}[section]
\newtheorem{lemma}{Lemma}[section]
\newtheorem{corollary}{Corollary}[section]
\theoremstyle{definition}
\newtheorem{definition}{Definition}[section]

\newcommand{\boldnabla}{\mbox{\boldmath$\nabla$}} % to be used in mathmode
\newcommand{\cbar}{\overline{\mathbb{C}}}% to be used in mathmode
\newcommand{\diff}[2]{\frac{d #1}{d #2}}% to be used in mathmode
\newcommand{\difff}[2]{\frac{d^2 #1}{d #2^2}}% to be used in mathmode
\newcommand{\pdiff}[2]{\frac{\partial #1}{\partial #2}} % to be used in mathmode
\newcommand{\pdifff}[2]{\frac{\partial^2 #1}{\partial #2^2}}% to be used in mathmode
\newcommand{\upperth}{$^{\mbox{\footnotesize{th}}}$}%to be used in text mode
\newcommand{\vect}[1]{\mathbf{#1}}% to be used in mathmode
\newcommand{\curl}[1]{\boldnabla \times \vect{#1}} % to be used in mathmode
\newcommand{\divr}[1]{\boldnabla \cdot \vect{#1}} %to be used in mathmode
\newcommand{\modu}[1]{\left| #1 \right|} %to be used in mathmode
\newcommand{\brak}[1]{\left( #1 \right)} % to be used in mathmode
\newcommand{\comm}[2]{\left[ #1 , #2 \right]} %to be used in mathmode
\newcommand{\dop}{\vect{d}} %to be used in mathmode
\newcommand{\cov}{\text{cov}} %to be used in mathmode
\newcommand{\var}{\text{var}} %to be used in mathmode
\newcommand{\mb}{\mathbf} %to be used in mathmode
\newcommand{\bs}{\boldsymbol} %to be used in mathmode
% Title Page
\title{How informative are retinal ganglion cells?}
\author{Thomas Delaney 1330432}

\begin{document}
\section*{Modelling higher-order correlations within cortical microcolumns}
\subsection*{Urs K\"{o}ster, Jascha Sohl-Dickstein, Charles M. Gray, Bruno A. Olshausen}
\subsubsection*{Abstract}
	Spiking activity is taken from a cortical micro-column and modelled to an Ising model, a restricted Botlzmann machine, and a semi-restricted Boltzmann machine. Model parameters were fitted using a method entitled \textit{minimum probability flow}. Log likelihoods were estimated using \textit{annealed importance sampling}. The hidden units in the RBM and sRBM reflected the laminar structure of the cortical micro-column. The RBM and sRBM both outperformed the Ising model.

\subsubsection*{Introduction}
	It was hoped that using an RBM would allow the hidden units to model higher order correlations than the Ising model. The hidden units reflected the layered structure of the cortical micro-column, showing that the layered structure is the main source of correlations in the visual cortex. 
	
\subsubsection*{Results}
	$L_1$ regularization was used on the parameters to avoid overfitting. This resulted in many of the hidden units having zero couplings.
	
\end{document} 