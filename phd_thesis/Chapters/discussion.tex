\chapter{Discussion}
Ideas (not in order):
\begin{itemize}
  % calcium
  \item We built the model, it's biophysical, Judging by Greenberg biophysical modelling is the way to go.
  \item how we fitted (amplitude and power spectrum), resulted in similar spike inference perf
  \item we investigated the effects of buffers, indicator properties
  \item investigated the modelled fluorescence trace of various firing rates including high ones
  \item our fluorescence model could be useful in a number of situations.
  \item voltage imaging
  % Functional networks
  \item Applied new network science to new electrophysiological dataset.
  \item applied clustering comparison measurements to assess similarity to anatomy
  \item time bin width has an effect on correlation size, and correlation distribution
  \item further differences in means, different when we take regions into account
  \item used new method to find dimensions of additional structure
  \item we detected communities, they resemble anatomical division at short timescales
  \item conditioned on video (mention and explain)
  \item different types of correlations
  \item Results might be intuitive, but are new nonetheless (as far as I know)
  \item Potential for more network science applications?
  % COMb
  \item Applied the COMB distribution to neuronal data for the first time.
  \item investigated changes in response to stimulus, different for different regions
  \item best fit
  \item captures changes better than correlations, offers a good alternative to correlation for quantifying association
  \item replicated stimulus related quenching
  \item captures correlated behaviour by quantifying \textit{association}.
  \item coupling with existing models could yield some nice models.
  \item More statistical invention could be useful. Conway-Maxwell-Poisson distribution
\end{itemize}

In this project, we attempted to address some of the challenges in data collection from large neuronal ensembles (specifically with calcium imaging) and some of the problems in analysing the data collected from large neuronal ensembles.

Firstly, we investigated the relationship between cell biochemistry, action potentials and the fluorescence traces produced by fluorescent calcium indicators. We did this by building a biophysical model that takes in a spike train and returns the fluorescence trace that that spike would induce. The model included mechanics for the binding of calcium to fluorescent and endogenous mobile and immobile buffers, and the consequent changes in concentration of free and bounded calcium. The model consisted of $17$ parameters, $13$ of which were set according to data from the literature, and $4$ of which were free parameters. We trained the model using simultaneously collected spiking and calcium imaging data \parencite{berens}. We fitted the model by matching the $\Delta F/F_0$ in response to an action potential, and by matching the power spectrum of the actual fluorescence trace. This meant that our model would include the correct amount of noise as well as return the correct change in amplitude in response to an action potential.

Since our model produced fluorescence traces, we could apply spike inference algorithms to the modelled fluorescence traces that our model produced after training, and compare the performance of the algorithms on the modelled traces to their performance on the real traces. We used three spike inference algorithms, two of which were based on modelling the calcium trace as an autoregression \parencite{friedrich, pnevmatikakis}, and another inference algorithm that was a little more biologically inspired, but amounted to a very similar algorithm \parencite{deneux}. We compared the performance of the algorithms by using them to infer spikes from $20$ real and modelled fluorescence traces induced by $20$ corresponding real spike trains. We then used several binary classification measures (true positive rate, accuracy etc.) to asses the quality of the spike inference for the real and modelled fluorescence traces. We found that ...
