\chapter{Introduction}

\label{chap:intro}

\section{Overview}
% Ideas (not in order):
% \begin{itemize}
%     \item From small to big datasets (in terms of number of neurons)
%     \item Big datasets mean statistical methods are more necessary (curse of dimensionality)
%     \item Big datasets mean higher order correlations are more meaningful (schneidman)
%     \item Exploit pairwise correlations in different way (eight probe)
%     \item abandon correlations embrace association (COMB)
%     \item electrophysiology drawbacks vs calcium benefits
%     \item calcium drawbacks (fluorescence modelling) (mention nuclear filling and cell pathology) (mention that calcium imaging can only be used near the surface of the brain, e-phys can go deeper, especially with new probes)
% \end{itemize}

Since Hodgkin and Huxley's squid experiments featuring a single axon \parencite{hodgkin}, to more recent research with spike sorted data from  $\sim 24000$ neurons from $34$ brain regions from $21$ mice \parencite{allen}, the number of neurons contributing to electrophysiological datasets has been growing. The number of simultaneously recorded neurons has doubled approximately every seven years since the use of multi-electrode recording in neuroscience began \parencite{stevenson}. Recording methods using two-photon calcium imaging have also been used to extract data from populations containing over $10000$ neurons \parencite{peron}. This dramatic growth in the number of neurons available for analysis requires a dramatic change in analysis methods.

There are multiple methods for reading activity from neuronal ensembles: electrophysiology, calcium imaging, and voltage imaging. Electrophysiology involves inserting electrodes into the brain of an animal. The electrodes read extra-cellular membrane potential, and using these readings we observe activity in the ensemble. Calcium imaging and voltage imaging use indicator dyes or fluorescent proteins that emit fluorescence traces that indicate either the concentration of calcium in a neuron's cytoplasm, or the neuron's membrane potential. In this project, we have attempted to address some of the difficulties in collecting data from these large ensembles using fluorescent calcium indicators, and some of the difficulties in analysing the collected data.

The rest of this introductory chapter will give some background about methods of recording from the brain, and some background for the rest of the document. Chapter two describes a biophysical model for the fluorescence trace induced by a given spike train in a cell containing a fluorescent calcium indicator. Our third chapter describes our investigations into the correlated activity across different regions of a mouse behaving spontaneously. We applied a novel community detection method \parencite{humphries} from network science to correlation based networks of neurons, and observed differences in the structure of these correlations at different timescales. In our fourth chapter, we detail a new statistical model for the number of neurons spiking in a neuronal ensemble at any given moment. With this model, we attempted to capture correlated activity in a new way. The fifth chapter is a brief description of the work that yielded negative results or was abandoned. The final chapter is a discussion of our work and results from the previous chapters and their implications.

\section{Modelling the fluorescence of calcium indicators}

To focus on calcium imaging for a start, a neuron that contains a fluorescent calcium indicator in its cytoplasm will fluoresce when bombarded with photons. The amount that the cell will fluoresce is dependent on the concentration of fluorescent indicator within the cell, and the concentration of calcium within the cell. When a neuron fires an action potential, the influx of free calcium ions causes an increase in fluorescence when those ions bond with the fluorescent indicator and those bounded molecules are bombarded with photons. After the action potential, as calcium is extruded from the cell the fluorescence returns to a baseline level. This is the basic mechanism of fluorescent calcium indicator based imaging.

This method has some advantages over electrophysiology as measure of neuronal ensemble activity. Many of the problems with electrophysiology are within the processes used to isolate spikes in the extracellular voltage readings, and assign these spikes to individual cells. These processes are collectively called `spike sorting'. A comparison of many different spike sorting algorithms found that these algorithms only agreed on a fraction of cases \parencite{buccino}. Furthermore, because electrodes measure extracellular voltage, neurons that do not spike will not be detected. Isolating individual neurons is easier and more reliable when using calcium imaging data, because cells will emit a baseline level of fluorescence when not firing action potentials. Another advantage is that calcium imaging sites can be re-used for weeks for longitudinal studies \parencite{chen}. One of the methods of delivering the fluorescent indicator is by adeno-associated viruses, consequently there can be problems with indicator gradients around the infection site, and expression levels will change in individual cells over weeks \parencite{tian, chen}. This delivery method can also cause cell pathology, and nuclear filling \parencite{zariwala}, but these problems can be solved by using lines of transgenic mice \parencite{dana}. The fluorescence signal itself can serve a a good indicator of cell activity, but similarly to electrophysiology, the aim of calcium imaging is often spike detection.

If the imaging data is collected at a high enough frequency, and the signal-to-noise ratio of the fluorescence trace is high enough, it should be possible to infer the spike times to some level of accuracy. For example, the calmodulin based indicator GCaMP6s has a sufficiently high signal-to-noise ratio that isolated action potentials can be detected and inferred \parencite{chen}. Many spike inference algorithms exist \parencite{vogelstein, pnevmatikakis, friedrich, paninski1, paninski2, deneux, greenberg}, and some of these can perform both cell isolation and spike detection simultaneously \parencite{vogelstein, pnevmatikakis, paninski2, deneux}. But the relationship between spiking and fluorescence change is not fully understood. For example, the fluorescent indicator will act like an additional calcium buffer within the cell cytoplasm and will compete with the other endogenous buffers to bind with free calcium ions. Therefore, the concentration of those endogenous buffers, and the binding dynamics of those buffers will have an effect on the change in fluorescence in response to an action potential. Furthermore, the binding dynamics of the fluorescent indicator itself will have an effect on the change in fluorescence. For example, the GCaMP series of fluorescence indicators are based on the calcium buffer protein calmodulin. This protein has four binding sites, whose affinities interact non-linearly. But most of the spike inference algorithms model the fluorescence as a linear function of a calcium trace, and they model this calcium trace as a first or second order autoregression with a pulse input to represent action potentials. Deneux et al. (2016) developed two different calcium fluorescence models behind their spike inference algorithm (MLspike) with a more biological inspiration. For their simpler model, they take a physiological approach and account for baseline calcium indicator dynamics. They end up with a system of first order differential equations defining the dynamics of calcium concentration, baseline fluorescence, and fluorescence. For their more complicated model specifically for genetically encoded calcium indicators, they also took into account indicator binding and unbinding rates, which added another equation to their system of equations. The algorithms that use the autoregression model and the MLspike algorithm are outperformed by the most recently published spike inference algorithm \parencite{greenberg}. This algorithm takes into account the binding dynamics of both the endogenous buffers and fluorescent calcium indicator, and the concentrations of free calcium, indicator, and endogenous buffer within the cell cytoplasm. The performance of this algorithm shows that there is value in more biologically inspired models of fluorescent calcium indicators.

In light of the growing popularity of two-photon calcium imaging, and the lack of biologically inspired spike inference algorithms (\parencite{greenberg} developed their spike inference algorithm in parallel to our work), we decided to develop a biologically inspired model for fluorescent calcium indicator fluorescence. The idea being that our model would take a spike train, or simply spike times, provided by the user, and return the fluorescence trace that would be induced by this spike train or spike times. The model contains parameters for concentrations of indicator and endogenous buffers, as well as affinity and unbinding rates for these buffers. There are also parameters for the baseline concentration of free calcium in the cell cytoplasm, and the cell radius (as a means for calculating the cell volume). With this model, we hoped that experimentalists would be able to test out different calcium indicators on the types of spike trains that they expect to encounter. This way they could decide ahead of time which indicator suited their situation best. Since the output of our model is a fluorescence trace, the spike inference models mentioned above can be applied to the modelled fluorescence. This means that the model could also be used to benchmark the performance of these spike inference algorithms, and to investigate the impact of variations in the model on spike inference accuracy.

\section{Functional networks}

We have outlined some of the advantages that calcium imaging has over electrophysiology. But electrophysiology is more useful in some situations. One particular drawback for two-photon calcium imaging is that usually it can only be used for imaging near to the surface of the brain. This problem can be solved by removing the tissue around the area to be imaged, and custom building a two-photon microscope \cite{dombeck}. Imaging with three (or presumably more) photons may solve this problem in the future \parencite{ouzounov}. A better option for reading activity from neurons beyond the surface of the brain is to use Neuropixels probes \parencite{jun}. These probes can be used to read from thousands of neurons simultaneously in many different areas of the brain \parencite{allen, stringer, steinmetz, steinmetz2019}. This brings us to another problem for which we require new innovations in our analysis methods. Specifically, analysing correlated behaviour in neural ensembles consisting of neurons from many different brain regions.

Until the invention of new technologies such as the Neuropixels probes, most electrophysiology datasets read from neurons in only one or two regions. Therefore most of the research on interactions between neurons in different regions is limited to two regions \parencite{wierzynski, patterson, girard}. In chapters \ref{chap:eight_probe} and \ref{chap:comb} we used datasets with neurons from $9$ and $5$ different brain regions respectively. In their review of the interaction between growing the number of neurons in datasets and the analysis methods applied to those dataset, Stevenson and Kording (2011) assert that an important objective of computational neuroscience is to find order in these kinds multi-neuron of datasets. This was our main aim for the research described in chapter \ref{chap:eight_probe}.

% might be asked to expand this paragraph, could explain more about the findingings in the citations.

In light of recent findings based on correlated behaviour showing that spontaneous behaviours explain activity in many different parts of the brain that would otherwise be regarded as noise \parencite{stringer}, that satiety is represented brain wide \parencite{allen}, and that exploratory and non-exploratory states are represented in the amygdala \parencite{grundemann}, it was clear that state representation or motor control had an influence on correlated behaviour in areas of the brain not usually associated with these tasks. Also, given differences in timescales of fluctuations in different brain regions \parencite{murray}, and different timescales for event representation in different brain regions \parencite{baldassano}, we decided to investigate brain wide correlated behaviour at timescales ranging from $5$ms up to $3$s.

We started off measuring the correlations in spike counts between individual neurons in our ensemble. These measurements induced a weighted undirected graph where each node represented a neuron, and the weight of each edge was the strength of the correlation between the neurons represented by the nodes at either end of that edge. In order to put the neurons into groups with correlated behaviour, we applied a novel community detection algorithm \parencite{humphries} to this graph. We repeated this analysis for timescales from milliseconds to seconds. Bear in mind that our correlation based graph was completely agnostic of the anatomical regions in which our cells resided. We then compared our correlated communities to their anatomy at each timescale. In this way, we used a novel method, never applied neuronal data before, to analyse the makeup of correlated communities across different regions at different timescales.

\section{A new statistical model for capturing correlated behaviour}

Many important findings have been made by measuring the correlations between binned spike counts, but there are some problems with this method of analysis. Firstly, the width of the bins used to bin spike times into spike counts has an effect on the magnitude of the correlations measured. Using a short bin width can cause your measurements to be artificially small \parencite{cohen2}. This may not be an issue if one is considering relative size of correlations when using the same bin width, but it is still not ideal. Secondly, while pairwise correlations can capture most of the information in a small network (up to $40$ cells) of highly correlated cells \parencite{schneidman}, a model based on pairwise correlations alone will fail to capture the activity of larger ($\sim 100$ cells) networks, higher order correlated activity is required \parencite{ganmor}. One problem with these higher order correlations is that they may be defined in different ways \parencite{staude}. Furthermore if we want to include them in a model this usually involves greatly increasing the number of parameters to fit, which increases the dimension of the parameter space leading to the `curse of dimensionality'.
Some models attempt to sidestep these problems while still capturing higher-order correlations. These models attempt to capture the relationship between each individual neuron in the ensemble, and the ensemble as a whole. Okun et al (2015) called the strength of this relationship the `population coupling', and demonstrated that this quantity can predict an individual neuron's response to optogenetic stimulation of the whole ensemble. They also showed that this quantity was an indicator of the neuron's synaptic connectivity \parencite{okun}. With the `population tracking model', O'Donnell et al. (2016) linked the probability of firing an action potential for each individual neuron with the distribution of the number of active neurons. This allowed model fitting for a large number of neurons, as well as calculation of full pattern probabilities, and population entropy \parencite{odonnell}.

In this work, we also aimed to capture correlated behaviour between the neurons in a neuronal ensemble without measuring correlations directly. Correlation coefficients capture the linear component of the relationship between two random variables, but will not measure any relationship beyond linearity. Also, measuring correlation coefficients using short timebins can be difficult for neuronal data \parencite{cohen2}. We decided to abandon correlation, and we aimed to quantify a more general concept of association by modelling the number of active neurons in the ensemble using a Conway-Maxwell-binomial (COMb) distribution \parencite{kadane_2016}.

The COMb distribution is a probability distribution over the number of successes in a sequence of Bernoulli trials, where these trials can be associated in some way. The COMb distribution is an extension of the standard binomial distribution, with an additional parameter to model association between the Bernoulli variables. Using this additional parameter the distribution can capture positive association, where the Bernoulli variables tend to take the same value, negative association, where the Bernoulli variables tend to take opposite values, or no association i.e. the standard binomial distribution.

We fit a COMb distribution to spike sorted electrophysiological data taken from five different regions in the brain of an awake mouse exposed to visual stimuli \parencite{steinmetz2019}. We examined whether or not a model based on the COMb distribution was able to capture changes in the number of active neurons in these neuronal ensembles in response to the stimuli. We also investigated the relationship between the changes as captured by the COMb model and the change in neural variability as measured by Churchland et al. in their famous paper \parencite{Churchland}.

Our overall aim was to investigate some of the challenges in analysing large ensembles of neurons present today. That included collecting the data to analyse (via calcium imaging), and subsequently analysing these data. We felt that this was a worthwhile project because the size of datasets, in terms of numbers of neurons and data collected, is growing rapidly. Consequently these challenges will only become greater unless they are addressed. This is our attempt at addressing them.
